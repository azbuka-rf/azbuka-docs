\documentclass[a4paper]{article}
\usepackage[14pt]{extsizes}
\usepackage[utf8]{inputenc}
\usepackage[russian]{babel}
\usepackage{setspace,amsmath}
\usepackage{epigraph}
% \usepackage{enumitem}
\usepackage{csquotes}
% \usepackage[unicode, pdftex]{hyperref}
\usepackage{amssymb}
\usepackage{amsthm}
\usepackage[top=2cm,bottom=2cm,left=3cm,right=2cm]{geometry}
\usepackage{indentfirst}
\linespread{1.3}

\begin{document}
  \thispagestyle{empty}
  \begin{center}
    Государственное бюджетное общеобразовательное учреждение Республики Мордовия\\\textbf{<<Республиканский лицей для одарённых детей>>}\\
    \hfill \break
    \hfill \break
    \hfill \break
    \hfill \break
    \hfill \break
    Проектная работа\\
    \textbf{Роботизированная шахматная доска\\<<Азбука>>}
  \end{center}
  \hfill \break
  \hfill \break
  \hfill \break
  \hfill \break
  \hfill \break
  \hfill \break
  \hfill \break
  \hfill \break
  \hfill \break
  \hfill \break
  \begin{tabular}{p{6.1cm}l}
     &\textbf{Выполнил:}\\
     &Тундыков Сергей Сергеевич\\
     &Ученик 9В класса\\
     &ГБОУ РМ <<Республиканский лицей>>\\
     & \\
     &\textbf{Руководитель:}\\
     &Кадикин Рушан Ринадович\\
     &Педагог доп. образования\\
     &ГБОУ РМ <<Республиканский лицей>>\\
  \end{tabular}
  \hfill \break
  \hfill \break
  \begin{center}
    г. Саранск, 2022 г.
  \end{center}
  \newpage
  \tableofcontents
  \newpage
  До карантина с другом любили играть в шахматы и тут капец. \\
  Но по интренету не те ощущения, по этому пришла идея фантомных шахмат \\
  Шахматы популярная игра \\ 
  Объект, предмет цели исследования \\ 
  Используемые материалы \\ 
  Экономическая оценка \\ 
  Оценка аналогов \\ 
  Экологическая оценка \\ 
  Применение \\ 
  Дальнеёшее развитие
  \newpage

  \section{Введение}
  Шахматы - игра увлекающая миллионы, берёт своё начало прямиком из Индии VI-VII века, за это время она конечно претерпела множество изменений, однако до сих пор остаётся одной из популярнейших 

игр. Я с моим другом тоже увлекаемся шахматами. Мы собирались, в парке, во дворах, на скамейках, у кого-нибудь дома практически каждый день, до тех пор пока не наступила пандемия. Самоизоляция 

нарушила устоявшиеся традиции и очные встречи стали не возможны. Сыграв несколько партий на онлайн сервисах я понял что это совсем не те ощущения, поэтому у меня родилась идея роботизированной 

шахматной доски.
  
  \section {Изучение аналогов}
  \begin{tabular}{|l|l|l|} \hline
    Название & Преимущества & Недостатки \\ \hline
  \end{tabular}

  \section{Процесс создания}
  \subsection{Первая модель}
  Создание проекта я начал со сборки электроники. На Arduino Mega я установил CNC Shield с двумя драйверами шаговых двигаетлей, подключил сами двигатели. Подобрал направляющие и другие части для 

картезианской кинематики перемещающей электромагнит. Далее откалибровал кинематику и понял что получилось слишком громоздко. Поэтому от дальнейшей разработки с этой кинематикой я отказался.
  \subsection{Вторая модель}
  После непродолжительных поисков в интернёте я обнаружил кинематику H-Bot, которая подходила для моего проекта. Я заново подбрал направляющие смоделировал и напечатал части для кинематики. 

Моделирование производилось в САПР <<Autodesk Fusion 360>>. Все это собрал и <<научился>> перемещать магнит от одной клетке до другой. При перемещении фигуры на доске было необходимо записать ход 

в написанную мной программу обрабатывающую ходы и получавшую ответные от известной шахматной программы Stockfish. Эта модель также получила корпус расчерченный в CorelDraw и изготовленный на 

фрезерном станке с ЧПУ из ДСП и ДВП.
  \subsection{Третья модель}
  В третьей модели я задумался над автоматическим распознаванием ходов на доске, для этого была изготовлена печатная плата на всё-том же фрезерном ЧПУ, на которую были смонтированы герконы. Так 

как добаыилась толщина платы и герконов пришлось увеличить толщину доски, что значило переделывать боковины.

  \section{Оценка проекта}
  \subsection{Экономическая оценка}
  При создании своего устройства я старался максимально удешевить его, при этом не жертвуя качеством изделия. Далее будет приведена стоимость компонентов, используемых при изготовлении проекта.

  \begin{tabular}{|l|l|l|l|} \hline
    Название & Кол-во & Цена за шт. & Итого \\ \hline
    Шаговый двигатель & 2 & 700руб & 1400руб \\ \hline
    ДСП 16мм & 1м^{2} & 250руб & 250руб\\ \hline
    ДВП 3мм & 1м^{2} & 100руб & 100руб \\ \hline
    Фанера 3мм & 1м^{2} & 150руб & 150руб \\ \hline
    Arduino Mega & 1 & 850руб& 850руб \\ \hline
    Грекон & 70 & 5руб& 350руб \\ \hline
    Wi-Fi модуль & 1 & 250руб & 250уб \\ \hline
    Блок питания & 1 & 600руб & 600руб
    Аренда обородования & & & 300руб
    3d печать & & & 300руб
    CNC Shield & 1 & 200руб & 200руб
    DRW8825 & 2 & 100руб & 200руб
    Расходные материалы & & & 200руб
    Направляющие 8мм & 4 & 100руб & 400руб
    Подшипник 13мм & 12 & 30 & 360руб
    Линейный подшипник & 4 & 120руб & 480руб
    
    
    
    Итого & & & 2281487руб \\ \hline
  \end{tabular}
  \subsection{Экологическая оценка}
  Корпус моего устройства изготовлен из ДСП и ДВП, которые можно полностью использовать вторично. Пластиковые части изготовленные на 3D принтере распечатаны из пластика поддающегося обработке, из 

чего следует, что при утилизации устройства он будет переработан в соответствии с ГОСТ Р 54259-2010. Утилизация электорники будет производится по ГОСТ Р 52106-2003. Технология изготовления, пир 

соблюдении техники безопасности и санитарно-гигиенических норм является безопасной для окружающей среды.
  \section{Результат работы}




\end{document}
