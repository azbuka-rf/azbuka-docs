\documentclass[a4paper]{article}
\usepackage[12pt]{extsizes}
\usepackage[utf8]{inputenc}
\usepackage[russian]{babel}
\usepackage{setspace,amsmath}
\usepackage{epigraph}
% \usepackage{enumitem}
\usepackage{csquotes}
% \usepackage[unicode, pdftex]{hyperref}
\usepackage{amssymb}
\usepackage{amsthm}
\usepackage[top=1.5cm,bottom=1.5cm,left=1.5cm,right=1.5cm]{geometry}
\usepackage{indentfirst}
\linespread{1.3}

\begin{document}
  Шахматы. Игра берёт свои начала в 16 веке в Индии, конечно с тех времён она претерпела множество изменений, однако до сих пор любима миллионами. Шахматы объединяют людей всех профессий и национальнсотей по всей планете: от лидера Острова свободы Фиделя Кастро, Владимира Ленина, Альберта Эйнштейна, Сергея Прокофьева, многих других и до Стэнли Кубрика и Арнольда Шварцнегера, который возможно сделал для популяризации шахмат больше чем многие гроссмейстеры. 
  
  Играть в шахматы любим и мы с другом. Мы играли, в парке, во дворах, на лавочках, когда было холодно приходили друг к другу в гости. Но эту идилию прервала пандемия коронавируса, когда была объявлена самоизоляция --- возможность очных встреч пропала. Первое что приходит на ум это множество сервисов онлайн шахмат, однако сыграв десяток-другой партеек, я понял что совсем не те ощущения, чем видя перед собой настоящую доску. И мне пришла идея роботизированной шахматной доски!

  Если вы когда нибудь занимались <<парным>> видом спорта, то знаете, что часто в секции набираются люди очень разного уровня, поэтому невозможно хорошо подобрать партнера каждому, с помощью моей роботизированной доски, можно подключить соперника похожего уровня из интернета с поддерживаемых платформ (пока это только LiChess) или подключая шахматные программы различного уровня поддерживающие Univesal Chess Interface (UCI).

  % Че нить о роборуках с картиночкой
  % Про секции (что противников хер подберёшь)
  % О личесс и стокфиш

  Я представляю вам мой проект --- роботизированная шахматная доска <<Азбука>>. Она может распозновать ходы, передавать их и получать ответные по Serial порту или по Wi-Fi. Под каждой клеткой поля установлен, так называемый геркон (Два контакта которые притягиваются внешним магнитом тем самым замыкая цепь в герметичной колбе, заполненой энертным газом или вакуумом  (Картиночка с герконом)). А соотвественно в каждой фигуре установлен небольшой магнит, который и замыкает геркон. Ходы соперника транслируются на доску с помощью электромагнита, расположенного под герконами.

  Электромагнит управляется драйвером \textit{A1488}, который управляет током на 12 Вольт получаемый с блока питания. Соленоид перемещается с помощью двух шаговых двигателей объединенных в так называемую кинематику H-Bot. Герконы смонтированы на печатной плате изготовленной на фрезерном ЧПУ-станке. Управляет доской платформа Arduino Mega с установленным на неё CNC Shield, двумя драйверами шаговых двигателей \textit{H44} и Wi-Fi модулем \textit{R3849}.

  Боковые стенки корпуса иготовлены из транспортной ДСП, а дно и <<крышка>> из ДВП обработанной как и боковины на фрезерного ЧПУ станке.

  В дальнейшем я планирую изменить микроконтроллер на esp-32, т.к. уже сейчас производительности arduino меня перестаёт устраивать, улучшить внешний вид: покрасить, скруглить углы, добавить функционал: например кнопки сдаться, предложить ничью, увеличить количество поддерживаемых онлайн платформ, добавить штатные шахматные часы, анализ партий, уменьшить толщину доски.
  


\end{document}